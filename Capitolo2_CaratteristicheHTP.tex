\documentclass[a4paper]{report}
\usepackage[T1]{fontenc}
\usepackage[utf8]{inputenc}
\usepackage[english]{babel}
\usepackage{amssymb}
\usepackage{textgreek}
\usepackage{graphicx}
\usepackage{multirow}
\usepackage{array}
\usepackage[a4paper,left=2cm,right=2cm]{geometry}
\usepackage{float}
\usepackage{subfig}
\usepackage{caption}
\usepackage{xcolor}
\usepackage{wrapfig}
\usepackage{amsmath}

\begin{document}
\tableofcontents
\chapter{Hydrogen Peroxide}

\section{History of Hydrogen Peroxide}

\texttt{\color{green} NUOVA VERSIONE}

Hydrogen peroxide is chemical compound which use had a great expansion mostly during the last century. It has been discovered in July, 1818 by Louis-Jacques Thenard and it was described as an "oxygenated water". Even if he was not the only nor the first chemist to manage this compound, he has been the first to identify it and to discover a production process. At the very beginning, he determined that the reaction between the barium peroxide and the nitric acid gave hydrogen peroxide (at low levels of purity), and, with the maturation of the experience, he found also that several acids and other alkali metal peroxides would form it. \\ 
Once discovered, it found its very first application in paper pulp production process and it has been the only one for almost fifty years, from the 1880 up to 1920 more or less, so it was produced in limited quantities. After 1925 the hydrogen peroxide request underwent a sharp increase, due to the fact that it started to be used also in the textile industry and it has been possible thanks to a modification of the productive process. In fact, the barium peroxide exploitation with hydrochloric acid was a large time demanding process, making it impossible to be adopted in large scale. In 1832 a variant of this procedure basing on the use of fluorosilic acid and not hydrochloric acid has been found and developed. With the emergence of these new productive processes, also the hydrogen peroxide concentration increased, passing from the 3\% in 1870 to 27.5\% in 1923 up to reach 35\% shortly after WWII, exploiting an electrolytic process.\\

\subsection{Hydrogen Peroxide as a Propellant}

The first major exploitation of hydrogen peroxide as a working fluid for propellant application is attributed to Hellmuth Walter for submarine turbine drive systems and assisted take-off units (ATO). In this period hydrogen peroxide was used only as a liquid monopropellant, in which a liquid catalyst were injected to dissociate it into hydrogen and oxygen, and then later devices would burn the oxygen to reach higher performances. \\ 
In this period it was produced exploiting three possible processes: 
\begin{itemize}
\item \textbf{Electrolytic Process:} starting from ammonium bisulfate through electrolysis peroxydisulfuric acid and hydrogen are produced. Than through hydrolysis of the latter produces again ammonium bisulfate and hydrogen peroxide, which is separated and concentrated by distillation.
\item \textbf{Anthraquinone Process:} an organic compound (anthraquinone) catalytically reacts with hydrogen. The catalyst is then separated from the reduced anthraquinone molecule and the compound is oxidized producing hydrogen peroxide, which is separated from the anthraquinone molecule and concentrated by distillation.  
\item \textbf{Oxidation of Propane:} propane of its derivatives are oxidized producing, after a series of passages, hydrogen peroxide.
\end{itemize}

The hydrogen peroxide which comes out of all these processes has a concentration of more or less 30\%, which is too low for propulsive applications. For this reason, further distillation processes are required, in order to increase it and make it a valid candidate. \\
Thanks to these post process distillations, in 1936 the German government was able to produce hydrogen peroxide with concentrations of 80-82\%, and, thanks to this possibility, H.Walter has been able to start its business firstly with the submarines turbine driver and then with the already-cited ATOs. \\
One of the most important historical applications of hydrogen peroxide by Germany was the V-2 turbopump gas generator, which used it as a monopropellant with a concentration of 80\%, where a liquid catalyst (potassium permanganate) was injected. The V-2 has been the precursor of all the ballistic missiles, used by Germany against Belgium and UK during the very last phases of WWII.\\

\texttt{\color{red} METTI IMMAGINE IMPIANTO V-2}

\subsection{Hydrogen Peroxide in the Post-War}

When WWII ended, the road opened by the V-2 in the study of the ballistic led to significant results also on torpedoes and rockets. Significant engines and developments have been encountered both in the post-war United Kingdom and in United States. \\

\subsubsection{United Kingdom}

At the very ending of the WWII, United Kingdom taken inspiration from the German engine designs and was also dependent from their 80\% hydrogen peroxide supply. After few years, they opened their own factories (like LaPorte, the current Solvay Interox) and started to produce by themselves a more pure and high concentration product, reaching the 85\%. Moreover, they started to project new aircraft rocket engines, completely different from the German ones. In 1952, the UK started the development of the Spectre, a HTP/kerosene aircraft rocket engine. In that case HTP was chosen thanks to logistical reasons, because HTP was easier to manage with reference to liquid oxygen. To be able to face any kind of inconvenience, for the Spectre project also another engine has been developed as a backup system, the Gamma 2, which has been the base for the Gamma 201 (the engine used on the first twelve Black Knight launches) and Gamma 301 (the engine used on the two last Black Knight launches). \\ 
The Black Knight has been probably the missile which had the highest performing hydrogen peroxide engine ever produced, using the propellant as a monopropellant pre-decomposed in a silver screen catalyst bed and the hot gases were after-burned with kerosene. \\
The Black Knight successor has been the Black Arrow, which was no longer a missile, but was a three stages rocket. The first two stages of it were driven by a liquid bi-propellant system, which used kerosene (RP-1) and HTP, while the third was driven by a solid propulsion system. During this phase a lot of rockets and engines were developed, basing on the couple RP-1/HTP with concentrations always higher than 80\%.\\

\subsubsection{United States}

Also in the United States the research and development on rocket and ballistic systems led to significant improvements, even if in this case they were moved by two other reasons: the cold war and the space race. \\
$H_2O_2$ played a significant role in this context, resulting as a successful propellant both in submarines and torpedo applications and aircraft engines, with the X-planes, which were able to break speed and altitude records by exploiting it. Unfortunately, the U.S. aim was basically different from the UK one. In fact, US had as principal objective demonstrate to be the most powerful country, so the performances were the most important aspect. For this reason, the most popular bi-propellant systems at that time were used to be based on liquid oxygen and nitrogen tetroxide. In this perspective, also the hydrogen peroxide rocket engines (AR2-3) developed in that period were mostly used to provide thrust augmentation for military aircraft. It had only experimental applications, like the LR-40 (its successor), so the massive production never started. \\
After that moment, hydrogen peroxide rocket engines progressively disappeared from the aeronautical and aerospace scenario, replaced by more performing devices. It lasted only for some submarines or torpedoes applications, as a turbo-pump gas generator. \\

\subsection{The Waning Success of Hydrogen Peroxide}

During the 1960-1970 the popularity of hydrogen peroxide progressively decreased due to a series of reasons. First of all, hydrazine began to play a significant role in the aerospace scenario. Even if it presented a series of problems at the very beginning (for example the need of an effective catalyst, of a refined propellant manufacture process and of a study on the material compatibility) once solved them, it performs much better than hydrogen peroxide as monopropellant and at the time performance was the most important feature of a propulsive system. In this perspective it was replaced also in bi-propellant applications by liquid oxygen, and in hypergolic systems by the couple nitrogen tetroxide-hydrazine. Hydrogen peroxide survived only in some niche applications that could not exploit the other compounds and at the end of '80s it was almost disappeared. \\ 
In addition to the performance reasons, a series of accidents, caused by a lack of experience, helped to give the idea that hydrogen peroxide was a dangerous and unstable compound.\\
In recent times, two incidents at the Stennis Space Center gave, above all the others, a negative perception of hydrogen peroxide. \\
The first one occurred during a fluid transfer maneuver: part of the system aimed to realize this operation was made with some incompatible materials. Since hydrogen peroxide decomposes if put in contact with incompatible materials (the decomposition rate depends on how much incompatible the materials are) and if the rid is not correctly vented, it is recommended to verify the affinity between the materials before use it. Unfortunately, at that time the hydrogen peroxide compatibility was not largely studied, and for this reason during that process an incompatible liner was used, causing in this way a failure due to over-pressurization. \\
Another accident that occurred in Stennis Space Center saw the destruction of a thrust chamber during the test of a hypergolic fuel with hydrogen peroxide. In general, to test hypergolic ignitions, it was common practice to use the oxidizer as the reference compound to determine the correct proportions between the two propellants, since the oxidizer is not usually the energetic material. Unfortunately this is not the case with hydrogen peroxide, because it can be used also as a monopropellant, and this oversight was the explosion reason, but for several years the responsability has been attributed to the danger of hydrogen peroxide. \\
These are the two most evident episodes, but during the studying phase a lot of mishaps occurred with hydrogen peroxide, and all of them contributed to create the idea that it was a dangerous compound up to the point that its use was largely not recommended for decades. \\
Only in recent years more in-depth studies have shown that a large numbers of casualties occurred due to incompatibility with the materials used or due to a significant lack of management experience, and a redemption process started. \\ 
%{CITA HISTORICAL HAZARDS}
\\
Starting in '90s, anyway, the need of a non-toxic, lower cost and environmental friendly propellant progressively increased, and this led to reconsider hydrogen peroxide as a valid substitute. In addition to that its ease of handling and relatively good performance helped on its affirmation.\\
At the moment it is largely studied for in-space applications on satellites and in the upper stages of space vehicles, in totally liquid or hybrid engines. \\

\section{Chemico-Physical Characteristics}

Hydrogen peroxide is a colorless liquid at room temperature (pale blue color), with a slightly sharp odor. \\
If pure hydrogen peroxide is considered, it has a boiling point at $423 K$ and a freezing point at $272.1 K$, which are different from the water properties, but not so much. In fact, since hydrogen peroxide is completely soluble in water, if diluted its concentration decreases and also the thermochemical properties become more similar to the water ones as can be seen in the following pictures. \\

\texttt{\color{red}METTI IMMAGINE BOILING/FREEZING}\\

In general, all the properties are influenced by the concentration so it is very important define it before to start the propulsive plant design because it affects both the density (which then influence all the combustion chamber dimensions) and the adiabatic decomposition temperature, which significantly increases with the concentration increase. \\%{CITA HYDROGEN PEROXIDE FOR PROPULSION AND POWER APPLICATIONS: A SWIFT PERSPECTIVE}

\begin{center}
\begin{tabular}{c c c}
\hline
weight \% $H_2O_2$ & Adiabatic decomposition temperature ($°C$) & density ($g/cm^3$)\\
\hline 
60 & 373 & 1.241\\
\hline 
70 & 233 & 1.289\\
\hline 
80 & 486 & 1.339\\
\hline 
90 & 740 & 1.392\\
\hline 
100 & 995 & 1.448\\
\hline
\end{tabular}
%\caption{Hydrogen Peroxide Characteristics at Different Concentrations} %{CITA HYDROGEN PEROXIDE FOR PROPULSION AND POWER APPLICATIONS: A SWIFT OF PERSPECTIVE}
\end{center}

\vspace{0.3 cm}
It is important to underline that its physical properties depend on concentration, so they may be similar to the water ones, but only at low concentrations grade. By increasing it, they may become very different.\\ 
In general, the two properties which are the most sensitive to hydrogen peroxide concentration are density (which can range from 1,009 $g/cm^3$ for a 3\% concentration up to 1,448 $g/cm^3$) and boiling/freezing point. In fact, as shown in the following picture, the boiling point increases up to reach 150 °C for pure hydrogen peroxide, while the freezing point reaches the minimum at -55 °C for a concentration around 60\%. \\ 
\texttt{\color{red}METTI IMMAGINE DENSITY}\\
\texttt{\color{red}METTI IMMAGINE BOILING/FREEZING}\\
From these two graphs it's possible to deduce that hydrogen peroxide is a storable compound which does not require any kind of cryogenic system.\\
In space propulsion applications hydrogen peroxide is used at very high concentrations (more or less at 87.5\% or, in rare cases, 98 \%) so, thanks to its great density, it allows to reduce all the propulsive plant volumes. Also at these concentrations, hydrogen peroxide remains a non-toxic, non-flammable and non-corrosive compound, does not react with the atmosphere and it is also compatible with a lot of pressurizing gases. Thanks to all of these features, to which its low vapor pressure (2 mm Hg) is added, this compound is considered very easy to handle. \\ 
The only lack in reference to hydrazine is about the stability. In fact, if in the latter no sign of dissociation processes is encountered even after decades, in hydrogen peroxide is not possible to evince the same. It requires the adding of a lot of stabilizers and the improvement of them lead to a dissociation ratio of  0.4\%/year. The dissociation ratio is also influenced by the materials with which it is in contact, so it is extremely important to choose some compatible materials.\\ %{CITA DEVELOPMENT OF HYDROGEN PEROXIDE MONOPROPELLANT ROCKETS}

\subsection{Decomposition of hydrogen peroxide}

Hydrogen peroxide is a very particular substance due to its unusual molecular structure. It consists in two atoms of hydrogen and two atoms of oxygen with an oxidation state of -1, which is uncommon because generally oxygen is encountered in oxidation state 0 or -2. This peculiarity allows to hydrogen peroxide to be used both as an oxidizing and as a reducing agent, depending on the pH of the solution in which it is used. Thanks to this characteristic, it can decompose via reaction of dispropornation as shown in the following picture.\\ %{CITA DECOMPOSITION OF HYDROGEN PERODIXE - KINETICS AND REVIEW OF CHOSEN CATALYSTS}

\texttt{\color{red} METTI IMMAGINE DISPROPORNATION}

\vspace{0.5 cm}
The reaction presented above is exothermic, which releases 94.6 kJ/mol if the water produced is in liquid state and 54.4 kJ/mol if the water released is in the form of steam, since the enthalpy of vaporization has to be considered. \\
This decomposition reaction occurs spontaneously in nature, but, as mentioned it is very slow with a rate of about 1\% per year for 95\% hydrogen peroxide.%{CITA SUTTON}
It could be accelerated by the external conditions like pressure or temperature, by the material with which it comes in contact, by the contamination degree of the latter, by the pH, the light exposition or surface movements. Generally, the materials which intentionally speed up the decomposition are called catalysts and they may be of two different species: homogeneous, that means that they are in the same physic state of the hydrogen peroxide (liquid), or heterogeneous, which constitute the majority of  the catalysts. \\
As rapidly mentioned, another way exploited to trigger the decomposition is by increasing the temperature. This second way is called thermal decomposition and it is a self-accelerating process if the heat is not properly dissipated.\\ 
Several studies have been conducted on hydrogen peroxide decomposition behavior and has been found that, using nitrogen as a environmental gas, hydrogen peroxide decomposes in two different ways depending on the temperature. More precisely, below 693K  the half-life time is independent on the hydrogen peroxide pressure, and predominates a heterogeneous first-order reaction, with an activation energy of 41,84 kJ/mol; on the contrary, above 693K it continues to be a first-order reaction, but the predominant behavior is heterogeneous with an activation energy of 200,83 kJ/mol. \\ %{CITA THE THERMAL DECOMPOSITION OF HYDROGEN PEROXIDE VAPOUR} 
During these experiments has been found that different external inert gases lead to different temperatures at which the hydrogen peroxide passes from an heterogeneous decomposition to an homogeneous decomposition, and also different concentrations of the inert gas lead to different temperatures and activation energies and, so, different rates of decomposition. 

\texttt{\color{red} METTI IMMAGINE HOMOGENEOUS AND HETEROGENEOUS REACTION 420°C}

\section{Hydrogen Peroxide: Monopropellant o Bi-propellant?}

\textbf{Bi-propellant}\\

Hydrogen peroxide can have interesting applications in bi-propellant systems. As a premise, it is important to specify that, generally, bi-propellant systems need a large amount of space required by the double number of feeding lines, tanks and turbopump pressurizing plant, when needed. For this reason, in general, hydrogen peroxide bi-propellant systems are used or on the rocket upper stages or in very large satellites, while in CubeSats it is preferable to use monopropellant solutions. \\
The use of hydrogen peroxide as an oxidizer starts long time ago with the first aviation engines, like V2, Redstone, Viking, Sea Slug and in the liquid rocket engine of the first step of Diamant B rocket in combination with UDMH (Unsymmetrical dimethylhydrazine) and in the first step of the Black Knight rocket in combination with kerosene. %{CITA Hydrogen Peroxide Decomposition Catalysts Used in Rocket Engines}
In recent times the research on hydrogen peroxide as an oxidizer is more oriented towards the reaction between hydrogen peroxide and hydrocarbons, because it has been demonstrated that this kind of reaction is also an environmental friend solution. Propellant assessment studies have demonstrated that kerosene and ethanol are the most promising candidates for the fuel role, even if they show some differences also in terms of performances. In particular, the match hydrogen peroxide/kerosene presents some advantages with respect with reference to hydrogen peroxide/ethanol; for example, kerosene requires a lower preignition pressure and temperature, and this reflects on the fact that also shorter time delays to reach autoignition conditions are required and on the fact that the steady-state combustion level can be obtained at lower combustion chamber pressures, leading to lower thermal loading on the chamber wall structure. %{CITA DEVELOPMENT OF A GREEN BIPROPELLANT HYDROGEN PEROXIDE THRUSTER FOR ATTITUDE CONTROL ON SATELLITES}
Also in terms of performances, kerosene seems to be a more valid candidate than ethanol. \\
Anyway, regardless the hydrocarbon chosen, at the moment the ignition can be performed without any kind of spark plug system: in fact is important to underline that ignition can be reached by the use of catalysts or hypergolically, with the addition of transition metal salts or with metal hydrides.\\ %{CITA ASSESSMENT OF VARIOUS FUEL ADDITIVES FOR RELIABLE HYPERGOLIC IGNITION WITH 98\%+ HTP}\
\\
\textbf{Monopropellant}\\

As mentioned, hydrogen peroxide can be used also as a monopropellant and this constitutes one of its point of interest thanks to the limited plant dimension required, making it attractive for CubeSats applications. \\ 
In this case, the hydrogen peroxide concentration has to be higher than 67\% due to the heat of evaporation of the products which is higher than the heat released by the decomposition. The higher the concentration, the better the performances.\\ %{CITA DEVELOPMENT OF A LIQUID BI-PROPELLANT ROCKET UTILIZING HYDROGEN PEROXIDE AS A MONOPROPELLANT}
Moreover, as widely demonstrated, the hydrogen peroxide decomposition produces only non hazardous and non polluting products, steam and oxygen, with a considerable amount of propulsive energy release.

\begin{equation}
2H_2O_2 (g) \rightleftharpoons 2H_2O (g) + O_2 (g)
\end{equation}

During this decomposition is it possible to reach temperatures up to more or less 1200K, which, despite being very high, allows to reach performances 20\% lower than the hydrazine ones. The specific impulse and the temperature are related by the following relation:

\begin{equation}
I_{sp}=K\sqrt{\frac{T_c}{\mu}}
\end{equation} 

where K is a proportional constant, $T_c$ is the combustion chamber temperature and $\mu$ is the products molecular weight. \\

\vspace{1cm}

Is it important to notice that in monopropellant cases hydrogen peroxide has to pass through a catalyst in order to be decomposed, but also in some bi-propellant solutions it has to be divided and then only oxygen ages like oxydizer, while the steam is just dispersed. In both cases, the catalyst has a fundamental importance and, for this reason, its characteristics have to be carefully chosen. \\ 

\section{Catalyst}

The catalyst choice is one of the most important problems in the development of engines based on hydrogen peroxide. Engines which exploit hydrogen peroxide decomposition, has the great advantage that their catalyst does not require any pre-heating phase, allowing the realization of s very simple engine. \\
Among the characteristics that it is possible to define during the selection phase particular relevance is represented by the material for all the catalyst parts, the shape of it and the optimal dimensions. The most important requirements that a catalyst should meet are: %{CITA HYDROGEN PEROXIDE DECOMPOSITION CATALYSTS USED IN ROCKET ENGINES}

\begin{itemize}
\item \textbf{High Activity:} the best catalyst possible is the one that minimizes the time between the moment in which hydrogen peroxide comes in contact with it and the moment in which hydrogen peroxide starts to decompose. \\
\item \textbf{Mechanical impact and thermal shock resistance:} when hydrogen peroxide starts to decompose there is a consistent release of heat, which has as a consequence an increase of temperature and, consequently an increase of pressure inside the decomposition chamber. For this reason, the catalyst is subjected to a consistent stress both mechanical and thermal.\\
\item \textbf{High activity and stability in a wide interval:} hydrogen peroxide can be found either in vapor phase either in liquid phase, so the catalyst should be designed in order to encounter both these operating conditions.\\
\item \textbf{Resistance to stabilizing agents:} in order to enhance the hydrogen peroxide long-term stability, some additives and stabilizers are added to it. These stabilizers degrade at high temperatures and they may poison the catalyst.\\
\item \textbf{Optimum ratio between the mechanical strength and specific surface area:} is important to find the best trade off between the catalyst porosity and the mechanical features. A low-porous catalyst has a low specific surface area, causing a very low decomposition ratio and leading to an incomplete decomposition. The main consequence of it is a consistent temperature decrease which can lead to the risk of catalyst drowning due to the fact that the decomposition products are not in gaseous form and the hydrogen peroxide is not fully decomposed. On the contrary, a high-porosity catalyst has a high specific surface area allowing, in this way, the total decomposition of hydrogen peroxide. The heat released is very high, causing a consistent temperature increase and a consequent huge pressure drop, which can cause the support disintegration. 
\end{itemize}

As a general rule, the best are the catalyst performances, the larger is the amount of hydrogen peroxide decomposed in the unit time and so the smaller the catalyst bed size can be. Clearly, the catalyst bed size should not be decreased excessively since it could lead to an excessive load on it, increasing the rate of the catalyst erosion. \\

\subsection{Types of catalyst}

The number of possible catalysts is almost unlimited, since they can differ both in shape and in materials for the active phase and the support, but is it possible to make a first preliminary big division: 
\begin{itemize}
\item{\textbf{Homogeneous catalysts:}} these are catalysts under fluid form, which mix with hydrogen peroxide flow. %{CITA MONOPROPELLANT THRUSTER DEVELOPMENT: INVESTIGATION OF DECOMPOSITION EFFICIENCIES} 
Generally, homogeneous catalysts are solutions containing $Fe^{2+}, Fe^{3+},Mn^{n+}$ or $Cu^{2+}$ %{CITA DEVELOPMENT AND GROUND TEST OF A 100-MILLINEWTON HYDROGEN PEROXIDE MONOPROPELLANT MICROTHRUSTER} 
and perform more violent catalytic reactions than heterogeneous ones. The choice of homogeneous catalyst has the great advantage of solving the catalyst degradation problem; but, on the contrary, it increases the global system complexity and mass, since it requires additional tank, valves and pipes.
\item{\textbf{Heterogeneous catalysts:}} these are catalysts under solid form, which take the name of \textit{catalyst beds}. This kind of solution allows to reach the maximum decomposition efficiency and has the minimum required volume. %{CITA MONOPROPELLANT THRUSTER DEVELOPMENT: INVESTIGATION OF DECOMPOSITION EFFICIENCIES} 
\end{itemize}

From the description presented above, it is possible to evince that for micro-satellites applications heterogeneous catalysts is the most suitable option. Among them, another division based on the shape is possible: it is common to encounter catalysts in the form of screens %{CITA HYDROGEN PEROXIDE ROCKET MANUAL}
, packed metal-oxide pellets %{CITA PULSE RESPONSE TIMES OF HYDROGEN PEROXIDE MONOPROPELLANT THRUSTERS}
, fine grains %{CITA SPACE PROPULSION ANALYSIS AND DESIGN}
, gauzes %{CITA DEVELOPMENT OF HYDROGEN PEROXIDE MONOPROPELLANT ROCKETS}
, porous foams %{CITA PROPERTIES OF CERAMIC FOAM CATALYST SUPPORTS: MASS AND HEAT TRANSFER}
and monoliths with flow channels. %{CITA EFFECTS OF CHANNEL GEOMETRY ON THE PERFORMANCE OF CATALYTIC MONOLYTHS}
\\
In hydrogen peroxide applications, the most successful solution is the pellet one, but in recent times monolithic catalysts are becoming more and more popular thank to the fact that they're going to solve one of the great problems of pellet: the great pressure drop due to the attrition between moving pellets, which also increases the thermal load. On the contrary, monolithic catalysts have a low pressure drop and good structural stability. In addition to that, they have a high controllability of the active surface area and a convenient manufacturing process. %{CITA MONOPROPELLANT THRUSTER DEVELOPMENT INVESTIGATION OF DECOMPOSITION EFFICIENCIES}
However, the low pressure drop enabled by the monolithic catalysts opens the doors to another problem: since the pressure drop straddling the catalyst is low, is it possible to reduce the pressure in the tanks and in the feeding lines ($\Delta p = 0.2-0.3 bar$), but this allows to some oscillations to go upstream and maybe couple with the natural frequencies of the feed lines. This kind of instability takes the name of \textit{chugging instability} and is one of the three possible instabilities inside the rocket engines, which involves the lowest frequencies. The other two are the \textit{buzzing} instability and \textit{screaming} instability, which involve respectively the intermediate frequencies and the higher frequencies. \\ %{CITA IL SUTTON}


\texttt{\color{red} METTI IMMAGINE engine instabilities-chugging}\\ 

\subsection{Catalytic bed sizing}

Before conducting a preliminary sizing of the catalyst bed is important to be conscious that a limit on the minimum dimension is present, independently by the shape chosen. Is it possible to determine this minimum dimension by a balance between the reaction rate and the heat loss rate of the system towards the external environment, which during ground tests is dominated by convection, while in in-space applications is dominated by radiation.  \\ %{CITA DEVELOPMENT AND GROUND TESTS OF A 100-MILLINEWTON HYDROGEN PEROXIDE MONOPROPELLANT MICROTHRUSTER}

\texttt{\color{red} INSERISCI IMMAGINE DIMENISONAMENTO CATALIZZATORE} \\

The energy balance of the system is the following: 

\begin{equation}
\dot{m}Q = \pi DLh \Delta T + \dot{m} \Delta H_{diff.}
\end{equation}

In the previous equation the left side term represents the energy generation by decomposition, the first right side term is the loss to the environment and the second term is the enthalpy convected by the stream. Moreover, in order to obtain an acceptable thrust level, is necessary that the system overcome the so called latent-heat barrier. To do that, it would be assumed that all the system is at a higher temperature than 423,15K, which is the hydrogen peroxide boiling temperature: in this way, when the hydrogen peroxide get in touch with the catalyst, it is already in a gaseous state, speeding up the decomposition process,which could be very low if the hydrogen peroxide remains in liquid state. \\ 
To obtain a self-sustained decomposition, is necessary that the heat generated is higher than the heat released towards the surrounding environment. Subtracting the left side term of the previous equation to the second term of the right side, and keeping in mind the last consideration, the following inequality is obtained: 

\begin{equation}
\pi Nu_dk_aL\Delta T \leq \dot{m}S_c
\end{equation} 

where $Nu_d$ is the Nusselt number of free convection, taken by an experimental formula of a horizontal constant temperature cylindrical tube proposed by Churchill |%{CITA “Correlating Equations for Laminar and Turbulent Free Convection from a Horizontal Cylinder,”}
, $k_a$ is the air conductivity and $S_c$ is the remnant reaction heat. Is important to underline that the time considered in the flux evaluation is the characteristic time, defined as the time required by the decomposition process in a catalyst bed.  As a simplification, the characteristic time is assumed to be equal to the residence time, which is the time spent by the liquid-phase hydrogen peroxide inside the combustion chamber. \\
Making explicit the $\dot{m}$ term in the previous equation: 

\begin{equation}
\pi Nu_dk_aL\Delta T \leq \rho\pi r^2 \frac{L}{t_c}S_c
\end{equation}

and replacing the constants with their real values, is it possible to trace a graph of the catalyst diameter with the residence time. \\

\texttt{\color{red} METTI IMMAGINE CATALYST BED DIAMETER VS RESIDENCE TIME}\\

Unfourtunately, when the diameter is so small, there is the onset of some instability phenomena which can lead to the crack of the channels. More in detail, if the temperature is higher than 473.15K, the gasification of hydrogen peroxide is very rapid, forming bubbles which can obstruct the channels and grow up to the channels breakdown. Then another bubble is formed, obstructing once again the channel and so on. This kind of instability is named as decomposition instability.\\ %{CITA DEVELOPMENT AND GROUND TESTS OF A 100-MILLINEWTON HYDROGEN PEROXIDE MONOPROPELLANT MICROTHRUSTER}
Also for this reason, in general, it would be better to have catalysts with diameter higher than 1 mm at least, which lead also to higher efficiencies.\\ It is important to add as a restriction that, in case of catalyst with a pellet geometry, the diameter dimension must be one order of magnitude smaller than the diameter of the combustion chamber.\\
For what concern the length of the catalyst, it is intrinsically dependent by the diameter chosen thanks to a fundamental parameter which characterizes the catalytic beds: the aspect ratio. This latter parameter is defined as: 

\begin{equation}
AR=\frac{L}{D}
\end{equation}

In general its value can vary in an interval from 0.2 and 2, but to have a good catalytic activity it should not be higher than 1. \\ %{CERCA E CITA L'ARTICOLO DA CUI HAI PRESO 'STI NUMERI}
Another important parameter for the characterization of catalyst beds is the \textit{Catalyst Bed Loading} (CBL), which is defined as the ratio of propellant mass flow rate and catalyst bed cross-sectional area.\\

\begin{equation}
G=\frac{\dot{m}}{A_c}
\end{equation}

In practical words, the catalyst bed loading is a parameter indicating how much hydrogen peroxide is decomposed by a known surface of catalyst. Clearly, fixing the CBL, is it possible to determine the diameter of the catalyst cross section, and this can be used as a check on the diameter which is obtained by the previous procedure. \\

\begin{equation}
G=\frac{\dot{m}}{\pi (\frac{d}{2})^2} \rightarrow
d=2\sqrt{\frac{\dot{m}}{\pi G}}
\end{equation}

From a series of practical experiments, has been found that a good catalytic activity is performed when the CBL varies in the range $G=[0.05;37.2]\frac{kg}{s \cdot m^2}$ . \\ %{CERCA E CITA L'ARTICOLO DA CUI HAI PRESO 'STI NUMERI}

\subsection{Popular catalyst materials}

When a catalyst is prepared, it is useful to remind that it is made by two different components which are made of two different materials: one is for the support, and the other one is for the active phase. \\
The only purpose of the support is to provide a help structure for the active phase retaining it by some adsorption reactions, but it does not take part in any way to the intermediate reactions. For this reason, the only constraints to which it is subject are the resistance to high temperatures and to mechanical stresses, which can be widely satisfied by ceramic materials, properly shaped. Among all the possible ceramic materials, the most common are the cordierite and the alumina, sometimes doped with silicium to delay the phase transition.\\
For what concerns the active phase, it can be composed by different materials, but in general metals are the most common choice. Here below the most popular and promising active phase are listed. 

\begin{itemize}
\item \textbf{Platinum:} It shows a very good activity, which depends however on the support material. In fact have been showed up that if it is loaded on alumina it has a catalytic activity one order of magnitude higher than the other ones, while if it is loaded on silica, its performances are comparable, or even lower, with reference to other active phases. Moreover, platinum has the huge drawback that it is a very expensive material, which limits its wide affirmation. %{CITA Performance of Different Catalysts Supported on Alumina Spheres for Hydrogen Peroxide Decomposition}
\item \textbf{Silver:} It is a widely used catalyst for several reactions and it is the oldest catalyst for hydrogen peroxide for space propulsion applications. It shows very good performances only under preheating conditions, which could increase the overall system complexity. Moreover it cannot withstand the high decomposition temperatures of hydrogen peroxide at concentration levels higher than 92\%, due to its low melting point. %{CITA HYDROGEN PEROXIDE GAS GENERATOR WITH DUAL CATALYTIC BEDS FOR NON PREHEATING STARTUP} 
From this brief overview is it possible to conclude that silver catalysts deteriorate very easily, so they are not the best possible choice under a cost/benefit point of view. 
\item \textbf{Potassium permanganate:} It shows almost the best performances, but its huge lack is that it does not adhere perfectly to the ceramic support and so it risks to be washed away by highly pressurized hydrogen peroxide. %{CITA HYDROGEN PEROXIDE GAS GENERATOR WITH DUAL CATALYTIC BEDS FOR NON PREHEATING STARTUP} 
In general permanganates are used only as a precursor for manganese oxides, not as an active phase themselves. 
\item \textbf{Manganese oxide}: It is the one of the most active catalysts in promoting the decomposition of hydrogen peroxide in the range of pH 4.5-7; moreover it is a very stable catalyst, thanks to its strong interaction with alumina surface. An additional advantage of this kind of catalyst is that is very cheap so, conjugated to its performances, this is the reason of its wide success.
\end{itemize}

\section{How to Manage Hydrogen Peroxide}

In this paragraph will be presented the measures that have to be adopted at all levels in order to prevent serious damages or injuries to stuffs or people during the managing of hydrogen peroxide. 

\subsection{Effect on humans}

Hydrogen peroxide is, generally, a non-hazardous compound, but it has to be managed correctly and carefully because, like all the chemical compounds, it could be seriously damage the human body. \\
Also at very low concentrations, if it comes in contact with the skin it could cause serious irritations of burn blisters even for brief exposure times like, for example, for bleaching effects. Prolonged exposure can result in thermal and/or chemical burns and if the clothes soak with hydrogen peroxide they have to be immediately removed and cleaned only with water. Also if the eyes come in contact with hydrogen peroxide they could suffer serious damages, sometimes also delayed in time, showing up only after some days. If the concentrations are very high and the exposure time is prolonged, in the worst cases it could lead to blindness. \\
If vapors of hydrogen peroxide are inhaled they could irritate and inflame the mucous membranes of the nose and throat. For this reason, a limit for the operators environment has been put to 1.4 $mg/m^3$ in air, for a worker which has to deal with it 8 hours/day and 40 hours/week. Analogously, in case of ingestion, severe irritations in mouth, throat, esophagus and stomach could be encountered, which could also be fatal. \\
In all these cases, the first thing to do is to try to dilute hydrogen peroxide trying to washing it away in case of skin/eye contact or drinking a large amount of water in case if ingestion. If inhaled, is important to move away and go outdoors to breath fresh air.\\

\subsection{Decomposition effects}

Hydrogen peroxide is a compound which spontaneously decomposes into steam and oxygen. Since its decomposition rate can be increased by several factor, there are some cautions to take during the handling and storage phases to prevent any kind of accidents. Moreover, in a preventing perspective, it is important to take some cares also during the storage plant design. \\
Decomposition rate of hydrogen peroxide is particularly sensitive to four categories of external disturbances: 
\begin{itemize}
\item \textbf{pH:} in alkaline solutions, so in mixtures which the overall pH is higher than 7, the decomposition rate of hydrogen peroxide is increased. This means that the pH affects deeply the stability of hydrogen peroxide, leading to an increasing loss of $H_2O_2$. For this reason, hydrogen peroxide and alkali should never be mixed up.\\
\texttt{\color{red} METTI IMMAGINE PH VS LOSS H2O2}\\
\item \textbf{Light:} exposition to light triggers a photo-chemical decomposition phenomenon, which leads to a significant increase in the decomposition rate, especially if the hydrogen peroxide is irradiated by sunlight. For this reason, and also for the one is following, the complete exposition to sun or, more generally, to heat sources should be avoided.
\item \textbf{Heat:} has been proved that hydrogen peroxide decomposition rate is more or less duplicated for each 10°C increase. More precisely, homogeneous decomposition rate can increase of two or three times, while heterogeneous decomposition can increase of one or two times. Since there is some release of heat during the decomposition reaction, if it is not properly dissipated a self-accelerating reaction may occur.
\item \textbf{Contaminants:} hydrogen peroxide is a compound that is very sensitive to the materials it comes into contact with, and it is incompatible with a lot of them. It could react very violently with a lot of metals (like iron, brass, copper, chromium, vanadium, platinum, silver etc..) with which it may come into contact in two ways: or these are the materials which the whole plant is made of, or these materials come accidentally in contact with hydrogen peroxide. Moreover, hydrogen peroxide results very sensitive to organic materials up to the point of creating some explosive peroxides, even if hydrogen peroxide itself is not explosive. 
\end{itemize}

In order to prevent any kind of issue, some measures are commonly adopted which can range from procedural measures up to design restrictions. \\

\subsection{Storage design}

When hydrogen peroxide decomposes, steam and oxygen in gaseous phase are produced. This means that the pressure inside the tank will slightly, but progressively, increase during time. This happens always, also when the decomposition process is not promoted by any of the effects listed above, since the spontaneous decomposition rate is 1\%/year. In order to avoid the burst pressure, is important to guarantee a correct venting and avoid to store it in hermetically sealed containers. \\
To meet the thermal issues, is important to guarantee that the heat produced by the decomposition process is correctly dissipated. In order to do that, the tanks have to be made of compatible and adapt materials, or, in the worst cases, they have to be cooled down by some devices responsible for doing it, for example some spraying systems of cold water. For this reason, also the ambient in which the tanks are conserved is usually not heated, unless exceptional cases, and vented. Moreover, all the conservation area has to be kept clean and free by flammable or incompatible materials, due to the possibility of leakages or drops. In this perspective, is also preferable to have the connecting pipes outside the conservation area or, at least, far away from any incompatible material. They have to be self-draining and the junctions have to be flanged, preferably avoiding threads and socket welds, because the latter could not properly prevent leaking or drops. Moreover, inside the plant have to be avoided any kind of pump which could potentially trap part of hydrogen peroxide inside of it, but to prefer for example centrifugal or gear pumps. In addition to that, also the number of the valves has to be minimized for two reasons: one is related to the pressure drop, which could become very large if a high number of valves is present, and so the plant would require an additional pump to compensate this drop; the other one is related to the fact that some valves require lubrication, which is not allowed in a plant devoted to hydrogen peroxide because lubricating oils are not compatible with hydrogen peroxide. The same concept is applicable to gaskets, which have to be avoided in general since are made by plastic materials, but mainly they have to be not lubricated. The last design constraints that have to be adopted in hydrogen peroxide plants concerns the filters: they should be strictly avoided, because in correspondence of the filters there is accumulation of dirt and impurities, which could interact with the compound making the whole system very dangerous. \\

\subsection{Handling Procedures}

In addition to the restrictions adopted during the project phase, there are also some measures that is useful to adopt during the handling phase. First of all, as a general rule, hydrogen peroxide should be kept in it original container without any kind of transferring to another one. In case it is not avoidable, HTP cannot be re-transferred in the first container, because it could be contaminated. In the same perspective, once a container of hydrogen peroxide has been emptied, it cannot be filled with another component if previously it has not been washed with distilled water and let to dry. Moreover, the discharging should be realized through siphoning, pumping or pouring and not by pressurization.\\
 Before using hydrogen peroxide in a rig or a plant is recommended to check that all the surfaces are not porous and then, in order to guarantee the safety, clean them up and passivate them. Passivation is a process which firstly remove all the impurities and contaminants from the plant surfaces (pickling), and then creates a layer of metal oxide which is compatible with hydrogen peroxide, but non-reacting. In this way, the purity of hydrogen peroxide is guaranteed in all the phases, since it comes in contact with non-reacting surfaces, and so also the stability and quality are preserved. \\
Also the tanks in which hydrogen peroxide is simply stored have to be passivated first, and this is why it would be better to leave the compound in the original contained and not move it to another one. Moreover, if hydrogen peroxide is stored, it would be a good practice to have retaining walls or bunds around the tanks, in order to catch and contain at least the entire amount of compound in case of rupture or leakage. Furthermore, close to the area interested by operations concerning hydrogen peroxide, it would be better to have a security shower or sink, to be ready to dilute it in emergency situations. \\



































\pagebreak



\texttt{\color{green} VECCHIA VERSIONE}




\section{Chemico-Physical Caracteristics \texttt{\color{red}VECCHIA VERSIONE}}

As previously said, hydrogen peroxide is a very flexible compound which can be used both in monopropellant systems and as an oxydizer in bipropellant systems. Regardless of how it is used, it has some attractive features which make it a valuable candidate for space propulsion. First of all, it has physical properties which are very similar to water ones, with the exception for the density, which is higher with reference to water even if it depends on the hydrogen peroxide concentration, and vapor pressure, which is lower than water. Thanks to its higher density, the $H_2O_2$ volumetric specific impulse is higher than hydrazine one. Moreover, its low freezing point and its high boiling point make it easy to handle and storage under standard conditions, since it remains in the liquid state at ambient pressure in a wide range of temperatures, and does not require any kind of cryogenic system, simplifying a lot all the rig. \\  %{CITA DEVELOPMENT OF HYDROGEN PEROXIDE MONOPROPELLANT ROCKETS}
Unfortunately, even if the volumetric specific impulse results higher than the hydrazine one, the specific impulse per se is more or less the 20\% lower than hydrazine, making it less performing and less attractive. In addition to that, also the hydrogen peroxide stability cannot be compared to the hydrazine one, since the latter can be loaded on satellites even for decades without encountering any kind of dissociation. On the contrary, hydrogen peroxide encounters a consistent dissociation process, mainly during the first years of use, reaching 1\%/year. Now, with the help of some stabilizers, it has been possible to decrease its dissociation ratio up to 0.4\%/year.\\ 
In reference to the other green propellants, hydrogen peroxide presents one great advantage which distinguishes it from all the others: its molecules are light, so the decomposition temperature is very low. This event allow to have a longer life of the combustion chamber and of the whole plant after the nozzle, and to perform a very large number of thermal cycles. \\
Moreover, this solution does not require any kind of pre-heating, making it simpler with reference to hydrazine systems; it has fast and repeatable performances and it is insensitive to poisoning. \\ 

\texttt{\color{red}METTI TABELLA CONFRONTO ACQUA, PEROSSIDO E IDRAZINA}




\end{document}